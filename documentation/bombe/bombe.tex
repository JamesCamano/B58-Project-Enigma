\documentclass[12pt,a4paper,fleqn]{article}
\usepackage[utf8]{inputenc}
\usepackage{amsmath}
\usepackage{algorithm2e}
\usepackage{amsfonts}
\usepackage{amssymb}
\usepackage[left=2cm,right=2cm,top=2cm,bottom=2cm]{geometry}
\author{James Camano}
\title{CSC B58: Bombe Machine Breakdown}

\setlength\parindent{0pt}
\newcommand{\lastedited}{(\textbf{Last Edited - \today{}})}

\begin{document}
	\maketitle
	\lastedited
	\begin{itemize}
		\item added components section
		\item added algorithm section
	\end{itemize}		
	
	This document sets out to describe the components of the \emph{Bombe machine} - the encryption setting-cracker counterpart to Enigma.
	
	\section{Introduction}
	Project Enigma sets out to imitate both the German \emph{Enigma} text cipher machine, and its mechanical counterpart - the Bombe machine.  \\
	
	This version of the Bombe machine will take advantage of the incremental patterns used by the Enigma machine, and will iteratively filter through potential solutions by method of contradiction.
	
	
	In this paper, we will assume the same definitions from the Enigma document.
	\section{Components}
	
	The Bombe machine consists of:
	
	\begin{itemize}
		\item A state counter, similar to that of Enigma's Rotor Component,
		\item A letter counter / incrementer circuit
		\item A checker circuit that uses simple lexicographic arithmetic to deduce contradictions.
	\end{itemize}		
	
	\section{Algorithm}

	Suppose that we are given the following sequence of letters $S_E $, outputted by Enigma:
	$S_E = \langle s_0, s_1, s_2, ..., s_n \rangle$, where the $s_i \in \Sigma$.
	
	To decrypt any message from Enigma, we need only look at the subsequence $\langle s_0, s_1, s_2 \rangle$, and determine a starting rotor configuration $R = (1, x)$ for which the following are true:
	
	\newcommand{\fhat}{\hat{f}}	
	
	\begin{flalign}
		\fhat(s_0, R) &= A  \\
		\fhat(s_1, g(R)) &= B  \\
		\fhat(s_2, g(g(R))) &= C
	\end{flalign}
	
	This comes from the fact that we have constrained any message of the enigma machine to be encrypted with the prefix ``ABC".
	
	With this in mind, the algorithm is: \newline
	
	\newcommand{\decodefn}{\textbf{decode}}	
	\newcommand{\inputseq}{\langle s_0, s_1, s_2 \rangle}
	\newcommand{\abcseq}{\langle A, B, C \rangle}
	\newcommand{\errorval}{11111}
	\newcommand{\lexshift}[1]{\textbf{lexshift(}#1\textbf{)}}
	\newcommand{\majorspace}{\BlankLine \BlankLine}
	
	\begin{algorithm}[H] \label{alg1highlevel}
	\newcommand{\rotorposn}{rotor\textunderscore{}position}
	\newcommand{\foundposn}{found\textunderscore{}correct\textunderscore{}position}
	
	\TitleOfAlgo{Bombe(S)}
	\KwIn{S: A message that is assumed to be encrypted by the Enigma machine.}
	\SetAlgoLined
	\KwResult{Finds and returns the correct starting rotor position of Enigma, if one exists.}
		% initialise the rotor position and the found position flag
		\rotorposn{}$:=0$\;
		\foundposn{}$:=false$\;
		$S := \inputseq$\; 
		
		\majorspace
		
		% loop as long as we haven't overflowed and we have not found correct rotor position.
		 \While{(\rotorposn\ $\neq 26$ \textbf{and not}(\foundposn))}{
		 	% if we pass S through decode and we get exactly ABC, then we know we have the sequence. 
		  	\eIf{(\textbf{\decodefn}(S[0:2], x) == $\abcseq$)}{
		   		\foundposn{}$:=true$\;
		   }{
				% or else, increment rotorposn
				\rotorposn{}++\;
		   }   
		 }
		 
		 \majorspace
		 % exiting here, we have a result
		 \eIf{(\foundposn)}{
		 	\KwRet{\rotorposn} \; % where the rotor position is our desired starting position
		 }
		 {
		 	\KwRet{$\errorval$} \; % where 31 is our ERROR value.
		 }
	 \caption{High-Level Algorithm for Bombe machine.}
\end{algorithm}

	\vspace{1cm}
	\begin{algorithm}[H] \label{alg2decode}
		\TitleOfAlgo{\decodefn(S, x)}
		\KwIn{S: The 3-letter required prefix for Encrypted messages from Enigma.}
		\SetAlgoLined
		\KwResult{Checks the given sequence $S = \inputseq$ and back-translates it. Returns whether or not the shifting value $x$ is the correct rotor starting position.}
		$s_0 = S[0]$\;
		$s_1 = S[1]$\;
		$s_2 = S[2]$\;
		
		\majorspace
		
		% note that we're shifting backwards
		$S^\prime := \langle \lexshift{s_0, -x}, \lexshift{s_1, -(x+1)}, \lexshift{s_2, -(x+2)} \rangle$\;
		
		\majorspace	
		
		\KwRet{$ S^\prime == \abcseq$}\;
		
		\majorspace
		\caption{The \decodefn(S, x) algorithm. Note that the function $\lexshift{s, x}$ represents a lexicographical shift in the direction of the sign of $x$. }
	\end{algorithm}

	\section{Justification for the Algorithm}
	
	\section{Additional Notes}	
	
\end{document}