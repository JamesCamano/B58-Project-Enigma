\documentclass[12pt,a4paper]{article}
\usepackage[utf8]{inputenc}
\usepackage{amsmath}
\usepackage{amsfonts}
\usepackage{amssymb}
\setlength\parindent{0pt}
\usepackage[left=2cm,right=2cm,top=2cm,bottom=2cm]{geometry}

\usepackage{blindtext}
\newcommand{\lastedited}{(\textbf{Last Edited - \today{}})}

\title{CSC B58: Enigma Breakdown}
\date{\today}
\author{James Camano}

\newcommand{\defn}[1]{\textbf{Def\textsuperscript{n}}:\ #1 - }
\newcommand{\validshifts}{\{i\}_{i=0}^{25}}

\begin{document}
	\maketitle
	\lastedited
	\begin{itemize}
		\item improved Rotor definition
		\item various function defintion improvements w.r.t. enc/dec algorithms
	\end{itemize}

	This document sets out to describe the components of the Enigma machine. \\
	\section{Introduction}
 	Project Enigma sets out to imitate both the German \emph{Enigma} text cipher machine, and its mechanical counterpart - the Bombe machine.  \\

	This imitation of Enigma (which will henceforth be called the same name) creates a cipher of a character input by starting off with an initial set state, performing \emph{alphabet shift arithmetic} to the character input based on that state and then `advancing' the state. Finally, this shifted input is returned as output.
	
	\section{Components}
	To describe the components of Enigma, we start with a few definitions. \\

	\defn{Rotor} In an Enigma machine, a Rotor $R$ can be described as an ordered pair $R = (n, m),  n \in \mathbb{N},  m \in \validshifts$. Where the value $n$ describes the rotor with relation to the other rotors in the same machine, and $m$ represents the value of the rotor. For example, if $R = (3, 15)$ then we say that $R$ is the third rotor, and its value is 15. \\

	Enigma consists of:

	\begin{enumerate}
		\item A set of rotors ${\{R^i\}}_{i=1}^{n}$ \footnote{Currently, $n=1$. }. 
	\end{enumerate}

	\section{Encryption Algorithm}
	We define:
		\begin{itemize}
			\item The alphabet $\Sigma = \{\bar{a}: \bar{a} \text{ is an uppercase character in the English alphabet} \}$
			\item $R_l$ to be a rotor with setting $n$. That is, $R_l = (n, m)$.
			\item $\varphi_k \in \Sigma$ to be the $k$\textsuperscript{th} letter in the alphabet. (e.g. $\varphi = B$ )
			\item $\omega_k$ to be the output letter corresponding to $\varphi_k$

			\item 
				$g(R_l) = \begin{cases}
					 (n, m+1), & \text{ if }  m+1 \leq 25 \cr
				        (n, 0), & \text{ if } {m+1} > 25 \cr 
				  \end{cases} 
				$
				
			\item 
					$f(\varphi_k, R_l) =	\begin{cases} 
						\varphi_{k}, & \text{ if } {k+m} \leq 25 \cr
						\varphi_{(k+m)-26}, & \text{ if }{k+m} > 25 \cr
					\end{cases}
					$ 
		\end{itemize}
	

	Then, the encryption algorithm is as follows:
	
	\begin{enumerate}
		\item $\omega_k := f(\varphi_k, R_n)$
		\item $R_k := g(R_k)$
	\end{enumerate}

	Where $\varphi_k$ is assumed to be the \emph{input} letter, and $\omega$ is the corresponding output of the Enigma machine. \\
	
	In words, given the input letter $\varphi_k$, Enigma shfits $\varphi_k$ the of $R_l$'s value positions, subsequently incrementing $R_l$ by one and then outputting the shifted letter $\omega$.

	\section{Decryption Algorithm}

	\newcommand{\fhat}{\hat{f}}
	Let the settings of the Enigma machine be similar to that in the encryption algorithm (i.e. If the starting position of R was $3$, then set R to $3$.) Then, define:

	 \begin{itemize}
		\item $\fhat(\varphi_k, R_l)= 					
					\begin{cases} 
						&\varphi_{k-m} \text{, if } {k-m} \geq 0 \\
						&\varphi_{(k-m)+26} \text{, if }{k+m} < 0
					\end{cases}
			  $ 
	\end{itemize}

	Then, the decryption algorithm is:
	

	\begin{enumerate}
		\item $\omega_k := \fhat(\varphi_k, R_l)$
		\item $R_n := g(R_l)$
	\end{enumerate}

	Where $\varphi_k$ is assumed to be the \emph{encrypted} letter, and $\omega_k$ is the corresponding original input of the Enigma machine (but $\omega_k$ is what is outputted). This method is intuitively straightforward since the subscript subtraction represents a shift of the same magnitude, but in the opposite direction. \\

	

	\section{Input Restrictions on Enigma}
	These restrictions are applied for the convenience of the Bombe machine.

	\begin{enumerate}
		\item Every input sequence to be encoded must start with the sequence $\langle A, B, C \rangle$.
		This is so that there is a definite (easy) flag for which the Bombe machine can deduce a contradiction.
		
	\end{enumerate}
	
			
	\section{Additional Notes}
		\begin{itemize}
			\item What will the output look like if we introduce a second rotor? A third?
			
			\begin{itemize}
				\item Shift terms will be of the form $x_1 + x_2 + r_1 + r_2$, where $x_1, x_2$ are the initial positions; $r_1, r_2$ are the number of shifts $\bmod{ 26}$ of the first and second rotor - respectively.
			\end{itemize}					
			
		\end{itemize}
\end{document}

