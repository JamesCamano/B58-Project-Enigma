\documentclass[12pt,a4paper]{article}
\usepackage[utf8]{inputenc}
\usepackage{amsmath}
\usepackage{amsfonts}
\usepackage{amssymb}
\setlength\parindent{0pt}
\usepackage[left=2cm,right=2cm,top=2cm,bottom=2cm]{geometry}

\usepackage{blindtext}
\newcommand{\lastedited}[1]{(\textbf{Last Edited - \today{}}: \emph{#1})}

\title{CSC B58: Enigma Breakdown}
\date{\today}
\author{James Camano}

\begin{document}
	\maketitle
	\lastedited{Began documentation of Enigma} \\ \\
	This document sets out to describe the components of the Enigma machine. \\
	\section{Introduction}
 	Project Enigma sets out to imitate both the German \emph{Enigma} text cipher machine.  \\

	This imitation of Enigma (which will henceforth be called the same name) creates a cipher of a character input by starting off with an initial set state, performing \emph{alphabet shift arithmetic} to the character input based on that state and then `advancing' the state. Finally, this shifted input is returned as output.
	
	\section{Components}
	Enigma consists of:

	\begin{enumerate}
		\item A set of rotors ${\{R^i\}}_{i=1}^{n}$, whose values cycle from $0-25$ \footnote{Currently, $n=1$. }. 
	\end{enumerate}

	\section{Encryption Algorithm}
	Define:
		\begin{itemize}
			\item The alphabet $\Sigma = \{\bar{a}: \bar{a} \text{ is a character in the English alphabet} \}$
			\item $R_n$ to be a rotor with setting $n$. That is, $R_n$'s value is $n$
			\item $\varphi_k \in \Sigma$ to be the $k$\textsuperscript{th} letter in the alphabet. (i.e. $\varphi_1 = b$ )

			\item $g(R_n) = \begin{cases}
					& R_{n+1} \text{, if } {n+1} \leq 25 \\
				      & R_0 \text{, if }{n+1} > 25
				  \end{cases} $
   
			
			\item $f(\varphi_k, R_n) =$ 
					$ 
					\begin{cases} 
						&\varphi_{k+n} \text{, if } {k+n} \leq 25 \\
						&\varphi_{k+n-26} \text{, if }{k+n} > 25
					\end{cases}
					$ 
		\end{itemize}
	

	Then, the encryption algorithm is as follows:
	
	\begin{enumerate}
		\item $\omega := f(\varphi_k, R_n)$
		\item $R_n := g(R_n)$
	\end{enumerate}

	Where $\varphi_k$ is assumed to be the input letter, and $\omega$ is the corresponding output of the Enigma machine. 

	\section{Decryption Algorithm}
	\blindtext
\end{document}

